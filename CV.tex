% -*- program: lualatex -*-
% !TEX program = lualatex
\documentclass[11pt,a4paper]{moderncv}

% microtype
\usepackage{microtype}

% Fix for Texlive 2017: https://tex.stackexchange.com/questions/373594/microtype-producing-dozens-of-unknown-slot-number-warnings?newreg=ded9fac9d8a747169d51bbacd8910d4e
\makeatletter
\def\MT@is@composite#1#2\relax{%
  \ifx\\#2\\\else
    \expandafter\def\expandafter\MT@char\expandafter{\csname\expandafter
                    \string\csname\MT@encoding\endcsname
                    \MT@detokenize@n{#1}-\MT@detokenize@n{#2}\endcsname}%
    % 3 lines added:
    \ifx\UnicodeEncodingName\@undefined\else
      \expandafter\expandafter\expandafter\MT@is@uni@comp\MT@char\iffontchar\else\fi\relax
    \fi
    \expandafter\expandafter\expandafter\MT@is@letter\MT@char\relax\relax
    \ifnum\MT@char@ < \z@
      \ifMT@xunicode
        \edef\MT@char{\MT@exp@two@c\MT@strip@prefix\meaning\MT@char>\relax}%
          \expandafter\MT@exp@two@c\expandafter\MT@is@charx\expandafter
            \MT@char\MT@charxstring\relax\relax\relax\relax\relax
      \fi
    \fi
  \fi
}
% new:
\def\MT@is@uni@comp#1\iffontchar#2\else#3\fi\relax{%
  \ifx\\#2\\\else\edef\MT@char{\iffontchar#2\fi}\fi
}
\makeatother

% French
\usepackage{polyglossia}
\setdefaultlanguage{french}

% Fonts
\usepackage{unicode-math}
\usepackage{fontspec}

\setmainfont[Ligatures={Common, TeX},
      Path           = fonts/SourceSansPro/,
      Extension      = .ttf,
      UprightFont    = *-Light,
      BoldFont       = *-Regular,
      ItalicFont     = *-LightItalic,
      BoldItalicFont = *-Italic]{SourceSansPro}

\newfontfamily\headingfont[Ligatures={Common, TeX},
      Path           = fonts/Alegreya/,
      Extension      = .ttf,
      UprightFont    = *-Regular,
      BoldFont       = *-Bold,
      ItalicFont     = *-Italic,
      BoldItalicFont = *-BoldItalic]{Alegreya}

% moderncv theme
\moderncvstyle{classic}
\moderncvcolor{red}
\moderncvicons{awesome}

% Style customization
%% Reduce name size
\renewcommand*{\namefont}{\fontsize{26}{28}\selectfont}

%% Change title font
\renewcommand*\titlefont{\LARGE\mdseries\slshape\headingfont}

%% Change section font
% \renewcommand*\sectionfont{\Large\headingfont}

%% Tweak colors

%% Text color

%%% Default color
\definecolor{color0}{RGB}{33,33,33}

% \definecolor{color0}{H}{757ce8}


%% Highligh color

%%% Teal
% \definecolor{color1}{HTML}{009688}

%%% Indigo
% \definecolor{color1}{HTML}{3F51B5}

\definecolor{color1}{HTML}{0B486B}
% \definecolor{color1}{HTML}{3B8686}

%%% Default blue
% \definecolor{color1}{RGB}{41,128,185}

%% Seconday text
% \definecolor{color2}{RGB}{117,117,117}
% \definecolor{color2}{RGB}{155,155,155}
% \definecolor{color2}{HTML}{607D8B}

% \definecolor{color2}{HTML}{3B8686}
\definecolor{color2}{HTML}{556270}

%% Reduce margins
\usepackage[top=0.8cm, bottom=1cm, left=1.5cm, right=1.5cm]{geometry}

%% Link with icon
\newcommand{\external}[2]{\href{#1}{{\tiny\faicon{external-link}}\,#2}}

%% Date column width
\setlength{\hintscolumnwidth}{2.5cm}

%% More packages
\usepackage{textcomp}
\usepackage[binary-units,locale=FR]{siunitx}
\DeclareSIUnit\octet{o}

%% Tilde
\AtBeginDocument{
  \renewcommand{\tilde}{{\raise.17ex\hbox{\ensuremath{\scriptstyle\sim}}}}
}

% Personal data
\name{Sébastien}{Brochet}

\title{Data Scientist \newline {%
\fontsize{14pt}{16pt}\selectfont Analyse big data -- Machine learning -- Développement%
}}

\address{Place des Martyrs 11}{1000 Bruxelles --- Belgique}{}
\phone[mobile]{+33~6~88~37~93~56}
\email{brochet.sebastien@gmail.com}
\social[linkedin][www.linkedin.com/pub/s\%C3\%A9bastien-brochet/77/577/6a9]{sébastien-brochet}
\social[github]{blinkseb}

\photo[70pt][0.4pt]{Photo.jpg}

\extrainfo{Disponibilité immédiate~\bullet~Permis B\\Mobilité région lyonnaise}

\begin{document}

  \makecvtitle%

  \vspace{-13mm}

  \section{Expériences professionelles}
    \cventry{mars 2015 --- en cours}
      {Chercheur en physique des particules (analyse de données)}
      {CP3 --- Université Catholique de Louvain}
      {Louvain-la-Neuve, Belgique}
      {}
      { \begin{itemize}%
          \item \textbf{Analyse de données (big data)} 2015 et 2016 de CMS au  LHC, CERN ($> \SI{50}{\peta\octet}$ de données)
          \begin{itemize}%
            \item Développement d'outils d'analyses innovant (simulations numériques et calculs scientifiques, C++ / Python).
            \item \textbf{Machine learning}: \textbf{deep neural networks} paramétriques avec \textbf{TensorFlow}.
            \item Traitement, modélisation et représentation des données.
            \item Analyse \textbf{statistique} des résultats: fit, p-valeurs, intervalles de confiance, limites d'exclusions.
            \item Résultats \textbf{approuvés} par la collaboration CMS et \external{http://cds.cern.ch/record/2257068}{\textbf{publiés}}.
          \end{itemize}%
          \item Développement et maintenance d'un \textbf{framework d'analyse de données} en \textbf{C++} utilisé par environ 30 personnes.
          \item Création, développement et support d'un outil scientifique en \textbf{C++}: \external{https://momemta.github.io}{MoMEMta} (\external{https://github.com/MoMEMta/MoMEMta/}{open source})%
          \begin{itemize}
            \item Architecture modulaire (via plugins) avec configuration en Lua (interfacage Lua / C++).
            \item Mise en place d'un \textbf{serveur d'intégration continue} avec Jenkins (tests unitaires et d'intégrations).
          \end{itemize}
          \item Environnement de travail \textbf{international}, communication \textbf{exclusivement en anglais}. Présentations régulières en groupe de travail (\tilde50 personnes).
        \end{itemize}
       }
    \cventry{oct. 2011 --- sept. 2014}
      {Doctorant en physique des particules en collaboration avec le \emph{CERN}}
      {\newline{} Institut de physique nucléaire de Lyon}
      {}
      {\small \newline{}Intitulé: \emph{Recherche de nouvelle physique dans le secteur du quark top avec l'expérience CMS au LHC}}{%
        \begin{itemize}%
          \item \textbf{Analyse des données} 2010, 2011 et 2012 du LHC\@.
          \begin{itemize}
            \item Création et développement de nouveaux outils d'analyse et de calcul scientifique (C++ et Python).
            \item Interprétation statistique des résultats.
          \end{itemize}
          \item Résultats \textbf{approuvés} par la collaboration et \textbf{publiés dans un journal renomé} (\emph{Physics Review Letters}).
          \item \textbf{Conférences internationales}: présentation des nouveaux résultats au nom de la collaboration CMS (Lake Louise, Canada, \tilde100 personnes).
          \item \textbf{Enseignement}: chargé de TP et TD niveau licence 1\iere{} et 2\ieme{} année (\tilde 200h, C++ et physique).
        \end{itemize}%
      }

  \section{Compétences}
    \cvitem{Scientifique}{Analyse et représentation des données, big data (data mining), machine learning, simulations numériques, statistiques}
    \cvitem{Programmation}{Connaissances approfondies en \textbf{C/C++}, \textbf{Python}, \textbf{Java} et \textbf{Bash}.\newline{}Maîtrise de Fortran, HTML/CSS, Javascript, \LaTeX, PHP}
    \cvitem{Systèmes}{Maîtrise avancée de Windows et Linux (Debian / Red Hat, desktop et serveur)}
    \cvitem{Frameworks}{Spark (SQL \& MLlib), matplotlib, scipy, TensorFlow, Keras, numpy, ROOT}
    \cvitem{SGBD}{MySQL, noSQL (MongoDB, Cassandra)}
    \cvitem{VCS}{\textbf{\external{https://github.com/blinkseb}{git}}, mercurial, bazaar, CVS, SVN}
    \cvitem{\faicon{flag}}{\textbf{Anglais courant} (oral et écrit, communication scientifique et rédaction d'articles)}

  \section{Formations}
    \cventry{2011 --- 2014}{Doctorat spécialité Physique des Particules}{Institut de Physique Nucléaire}{Lyon}{}{}
    \cventry{2009 --- 2011}{Master Physique}{Université Claude Bernard Lyon 1}{Lyon}{\textit{Mention TB}}{}
    \cventry{2005 --- 2009}{Licence Physique-Chimie}{Université de Franche-Comté}{Besançon}{}{}

  \section{Loisirs et activités}
    \cvitem{Développement}{\textbf{Co-créateur et développeur principal} de \external{http://www.androirc.com}{AndroIRC}, un client IRC pour \textbf{Android} (Java) ; maintenance des services web associés (serveurs + machines virtuelles)}
    \cvitem{Musique}{Pratique de la guitare dans un groupe depuis plus de 10 ans}
\end{document}
